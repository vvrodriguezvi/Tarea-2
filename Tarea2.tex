% Options for packages loaded elsewhere
\PassOptionsToPackage{unicode}{hyperref}
\PassOptionsToPackage{hyphens}{url}
%
\documentclass[
]{article}
\title{Tarea 2 - Análisis de Supervivencia}
\author{}
\date{\vspace{-2.5em}}

\usepackage{amsmath,amssymb}
\usepackage{lmodern}
\usepackage{iftex}
\ifPDFTeX
  \usepackage[T1]{fontenc}
  \usepackage[utf8]{inputenc}
  \usepackage{textcomp} % provide euro and other symbols
\else % if luatex or xetex
  \usepackage{unicode-math}
  \defaultfontfeatures{Scale=MatchLowercase}
  \defaultfontfeatures[\rmfamily]{Ligatures=TeX,Scale=1}
\fi
% Use upquote if available, for straight quotes in verbatim environments
\IfFileExists{upquote.sty}{\usepackage{upquote}}{}
\IfFileExists{microtype.sty}{% use microtype if available
  \usepackage[]{microtype}
  \UseMicrotypeSet[protrusion]{basicmath} % disable protrusion for tt fonts
}{}
\makeatletter
\@ifundefined{KOMAClassName}{% if non-KOMA class
  \IfFileExists{parskip.sty}{%
    \usepackage{parskip}
  }{% else
    \setlength{\parindent}{0pt}
    \setlength{\parskip}{6pt plus 2pt minus 1pt}}
}{% if KOMA class
  \KOMAoptions{parskip=half}}
\makeatother
\usepackage{xcolor}
\IfFileExists{xurl.sty}{\usepackage{xurl}}{} % add URL line breaks if available
\IfFileExists{bookmark.sty}{\usepackage{bookmark}}{\usepackage{hyperref}}
\hypersetup{
  pdftitle={Tarea 2 - Análisis de Supervivencia},
  hidelinks,
  pdfcreator={LaTeX via pandoc}}
\urlstyle{same} % disable monospaced font for URLs
\usepackage[margin=1in]{geometry}
\usepackage{color}
\usepackage{fancyvrb}
\newcommand{\VerbBar}{|}
\newcommand{\VERB}{\Verb[commandchars=\\\{\}]}
\DefineVerbatimEnvironment{Highlighting}{Verbatim}{commandchars=\\\{\}}
% Add ',fontsize=\small' for more characters per line
\usepackage{framed}
\definecolor{shadecolor}{RGB}{248,248,248}
\newenvironment{Shaded}{\begin{snugshade}}{\end{snugshade}}
\newcommand{\AlertTok}[1]{\textcolor[rgb]{0.94,0.16,0.16}{#1}}
\newcommand{\AnnotationTok}[1]{\textcolor[rgb]{0.56,0.35,0.01}{\textbf{\textit{#1}}}}
\newcommand{\AttributeTok}[1]{\textcolor[rgb]{0.77,0.63,0.00}{#1}}
\newcommand{\BaseNTok}[1]{\textcolor[rgb]{0.00,0.00,0.81}{#1}}
\newcommand{\BuiltInTok}[1]{#1}
\newcommand{\CharTok}[1]{\textcolor[rgb]{0.31,0.60,0.02}{#1}}
\newcommand{\CommentTok}[1]{\textcolor[rgb]{0.56,0.35,0.01}{\textit{#1}}}
\newcommand{\CommentVarTok}[1]{\textcolor[rgb]{0.56,0.35,0.01}{\textbf{\textit{#1}}}}
\newcommand{\ConstantTok}[1]{\textcolor[rgb]{0.00,0.00,0.00}{#1}}
\newcommand{\ControlFlowTok}[1]{\textcolor[rgb]{0.13,0.29,0.53}{\textbf{#1}}}
\newcommand{\DataTypeTok}[1]{\textcolor[rgb]{0.13,0.29,0.53}{#1}}
\newcommand{\DecValTok}[1]{\textcolor[rgb]{0.00,0.00,0.81}{#1}}
\newcommand{\DocumentationTok}[1]{\textcolor[rgb]{0.56,0.35,0.01}{\textbf{\textit{#1}}}}
\newcommand{\ErrorTok}[1]{\textcolor[rgb]{0.64,0.00,0.00}{\textbf{#1}}}
\newcommand{\ExtensionTok}[1]{#1}
\newcommand{\FloatTok}[1]{\textcolor[rgb]{0.00,0.00,0.81}{#1}}
\newcommand{\FunctionTok}[1]{\textcolor[rgb]{0.00,0.00,0.00}{#1}}
\newcommand{\ImportTok}[1]{#1}
\newcommand{\InformationTok}[1]{\textcolor[rgb]{0.56,0.35,0.01}{\textbf{\textit{#1}}}}
\newcommand{\KeywordTok}[1]{\textcolor[rgb]{0.13,0.29,0.53}{\textbf{#1}}}
\newcommand{\NormalTok}[1]{#1}
\newcommand{\OperatorTok}[1]{\textcolor[rgb]{0.81,0.36,0.00}{\textbf{#1}}}
\newcommand{\OtherTok}[1]{\textcolor[rgb]{0.56,0.35,0.01}{#1}}
\newcommand{\PreprocessorTok}[1]{\textcolor[rgb]{0.56,0.35,0.01}{\textit{#1}}}
\newcommand{\RegionMarkerTok}[1]{#1}
\newcommand{\SpecialCharTok}[1]{\textcolor[rgb]{0.00,0.00,0.00}{#1}}
\newcommand{\SpecialStringTok}[1]{\textcolor[rgb]{0.31,0.60,0.02}{#1}}
\newcommand{\StringTok}[1]{\textcolor[rgb]{0.31,0.60,0.02}{#1}}
\newcommand{\VariableTok}[1]{\textcolor[rgb]{0.00,0.00,0.00}{#1}}
\newcommand{\VerbatimStringTok}[1]{\textcolor[rgb]{0.31,0.60,0.02}{#1}}
\newcommand{\WarningTok}[1]{\textcolor[rgb]{0.56,0.35,0.01}{\textbf{\textit{#1}}}}
\usepackage{graphicx}
\makeatletter
\def\maxwidth{\ifdim\Gin@nat@width>\linewidth\linewidth\else\Gin@nat@width\fi}
\def\maxheight{\ifdim\Gin@nat@height>\textheight\textheight\else\Gin@nat@height\fi}
\makeatother
% Scale images if necessary, so that they will not overflow the page
% margins by default, and it is still possible to overwrite the defaults
% using explicit options in \includegraphics[width, height, ...]{}
\setkeys{Gin}{width=\maxwidth,height=\maxheight,keepaspectratio}
% Set default figure placement to htbp
\makeatletter
\def\fps@figure{htbp}
\makeatother
\setlength{\emergencystretch}{3em} % prevent overfull lines
\providecommand{\tightlist}{%
  \setlength{\itemsep}{0pt}\setlength{\parskip}{0pt}}
\setcounter{secnumdepth}{-\maxdimen} % remove section numbering
\usepackage{booktabs}
\usepackage{longtable}
\usepackage{array}
\usepackage{multirow}
\usepackage{wrapfig}
\usepackage{float}
\usepackage{colortbl}
\usepackage{pdflscape}
\usepackage{tabu}
\usepackage{threeparttable}
\usepackage{threeparttablex}
\usepackage[normalem]{ulem}
\usepackage{makecell}
\usepackage{xcolor}
\ifLuaTeX
  \usepackage{selnolig}  % disable illegal ligatures
\fi

\begin{document}
\maketitle

Se tiene una base de datos que proviene de un estudio de supervivencia
en mujeres con cáncer de ovarios sometidas a dos tratamientos distintos,
en se tienen en cuenta 7 variables que son: futime (tiempo de
supervivencia en días), fustat (estado de la censura donde fustat = 0 es
censurado, y = 1 es defunción, representa la falla), age (edad de la
paciente en años), resid.ds (persistencia de la enfermedad tras el
tratamiento 1 = No, 2 = Si), rx (tratamiento a la que es sometida la
paciente), ecog.ps (puntuación del test, ECOG = 1 es buen estado).

\includegraphics[width=1\linewidth,height=1.2\textheight]{Tarea2_files/figure-latex/unnamed-chunk-1-1}

\includegraphics[width=1\linewidth,height=1.2\textheight]{Tarea2_files/figure-latex/unnamed-chunk-2-1}

\includegraphics[width=1\linewidth,height=1.2\textheight]{Tarea2_files/figure-latex/unnamed-chunk-3-1}

\section{1. }

\textbf{Ajuste al menos tres modelos parametricos de la familia AFT y
seleccione uno de ellos de acuerdo a algún criterio o estadístico de los
vistos en clase. Inicialmente, debe usar todas las covariables, y
secuencialmente y con cada distribucíon, seleccione las variables
estad́ısticamente importantes de acuerdo a su valor-p.~Una vez seleccione
un modelo,evalúe los posibles efectos confusores de las variables que no
fueron incluidas en el modelo (si es que aplica) e interacciones.
INTERPRETE.}

\subsection{a. modelo exponencial}

Se ajusta un primer modelo con todas las covariables

\begin{Shaded}
\begin{Highlighting}[]
\CommentTok{\#Modelo con todas las variables}
\NormalTok{result1 }\OtherTok{\textless{}{-}} \FunctionTok{survreg}\NormalTok{(}\FunctionTok{Surv}\NormalTok{(futime, fustat) }\SpecialCharTok{\textasciitilde{}}\NormalTok{ age }\SpecialCharTok{+}\NormalTok{ resid.ds }\SpecialCharTok{+}\NormalTok{ rx }\SpecialCharTok{+}\NormalTok{ ecog.ps, }\AttributeTok{dist =} \StringTok{"exponential"}\NormalTok{, }\AttributeTok{data =}\NormalTok{ base) }

\FunctionTok{summary}\NormalTok{(result1)}
\end{Highlighting}
\end{Shaded}

\begin{verbatim}
## 
## Call:
## survreg(formula = Surv(futime, fustat) ~ age + resid.ds + rx + 
##     ecog.ps, data = base, dist = "exponential")
##               Value Std. Error     z       p
## (Intercept) 12.3913     2.0130  6.16 7.5e-10
## age         -0.0875     0.0338 -2.59  0.0096
## resid.ds2   -0.7659     0.7411 -1.03  0.3014
## rx2          0.6269     0.6162  1.02  0.3090
## ecog.ps2    -0.2523     0.6061 -0.42  0.6772
## 
## Scale fixed at 1 
## 
## Exponential distribution
## Loglik(model)= -90.6   Loglik(intercept only)= -98
##  Chisq= 14.78 on 4 degrees of freedom, p= 0.0052 
## Number of Newton-Raphson Iterations: 5 
## n= 26
\end{verbatim}

Se ajusta un nuevo modelo, pero esta vez sin la variable
\textbf{ecog.ps}

\begin{Shaded}
\begin{Highlighting}[]
\CommentTok{\#Modelo sin ecog.ps}
\NormalTok{result2 }\OtherTok{\textless{}{-}} \FunctionTok{survreg}\NormalTok{(}\FunctionTok{Surv}\NormalTok{(futime, fustat) }\SpecialCharTok{\textasciitilde{}}\NormalTok{ age }\SpecialCharTok{+}\NormalTok{ resid.ds }\SpecialCharTok{+}\NormalTok{ rx, }\AttributeTok{dist =} \StringTok{"exponential"}\NormalTok{, }\AttributeTok{data =}\NormalTok{ base)}
\FunctionTok{summary}\NormalTok{(result2)}
\end{Highlighting}
\end{Shaded}

\begin{verbatim}
## 
## Call:
## survreg(formula = Surv(futime, fustat) ~ age + resid.ds + rx, 
##     data = base, dist = "exponential")
##               Value Std. Error     z       p
## (Intercept) 12.3982     2.0401  6.08 1.2e-09
## age         -0.0906     0.0335 -2.70  0.0069
## resid.ds2   -0.6968     0.7296 -0.96  0.3395
## rx2          0.6157     0.6189  0.99  0.3198
## 
## Scale fixed at 1 
## 
## Exponential distribution
## Loglik(model)= -90.7   Loglik(intercept only)= -98
##  Chisq= 14.6 on 3 degrees of freedom, p= 0.0022 
## Number of Newton-Raphson Iterations: 5 
## n= 26
\end{verbatim}

Se ajusta un nuevo modelo sin la variable \textbf{rx}

\begin{Shaded}
\begin{Highlighting}[]
\CommentTok{\#Modelo sin rx}
\NormalTok{result3 }\OtherTok{\textless{}{-}} \FunctionTok{survreg}\NormalTok{(}\FunctionTok{Surv}\NormalTok{(futime, fustat) }\SpecialCharTok{\textasciitilde{}}\NormalTok{ age }\SpecialCharTok{+}\NormalTok{ resid.ds, }\AttributeTok{dist =} \StringTok{"exponential"}\NormalTok{, }\AttributeTok{data =}\NormalTok{ base)}
\FunctionTok{summary}\NormalTok{(result3)}
\end{Highlighting}
\end{Shaded}

\begin{verbatim}
## 
## Call:
## survreg(formula = Surv(futime, fustat) ~ age + resid.ds, data = base, 
##     dist = "exponential")
##              Value Std. Error     z       p
## (Intercept) 13.442      2.042  6.58 4.6e-11
## age         -0.102      0.035 -2.92  0.0035
## resid.ds2   -0.733      0.715 -1.02  0.3056
## 
## Scale fixed at 1 
## 
## Exponential distribution
## Loglik(model)= -91.2   Loglik(intercept only)= -98
##  Chisq= 13.62 on 2 degrees of freedom, p= 0.0011 
## Number of Newton-Raphson Iterations: 5 
## n= 26
\end{verbatim}

Se ajusta un nuevo modelo sin la variable \textbf{resid.ds}

\begin{Shaded}
\begin{Highlighting}[]
\CommentTok{\#Modelo sin rx}
\NormalTok{result4 }\OtherTok{\textless{}{-}} \FunctionTok{survreg}\NormalTok{(}\FunctionTok{Surv}\NormalTok{(futime, fustat) }\SpecialCharTok{\textasciitilde{}}\NormalTok{ age, }\AttributeTok{dist =} \StringTok{"exponential"}\NormalTok{, }\AttributeTok{data =}\NormalTok{ base)}
\FunctionTok{summary}\NormalTok{(result4)}
\end{Highlighting}
\end{Shaded}

\begin{verbatim}
## 
## Call:
## survreg(formula = Surv(futime, fustat) ~ age, data = base, dist = "exponential")
##               Value Std. Error     z       p
## (Intercept) 13.9339     2.1030  6.63 3.5e-11
## age         -0.1185     0.0339 -3.50 0.00047
## 
## Scale fixed at 1 
## 
## Exponential distribution
## Loglik(model)= -91.8   Loglik(intercept only)= -98
##  Chisq= 12.51 on 1 degrees of freedom, p= 0.00041 
## Number of Newton-Raphson Iterations: 5 
## n= 26
\end{verbatim}

\subsecion{b. distribución log-normal}

\begin{itemize}
\tightlist
\item
  Usando todas las variables
\end{itemize}

\begin{verbatim}
## 
## Call:
## survreg(formula = Surv(futime, fustat) ~ resid.ds + ecog.ps + 
##     rx + age, data = base, dist = "lognormal")
##               Value Std. Error     z       p
## (Intercept) 10.3932     0.9833 10.57 < 2e-16
## resid.ds2   -0.5331     0.3493 -1.53   0.127
## ecog.ps2    -0.0205     0.3203 -0.06   0.949
## rx2          0.6059     0.3027  2.00   0.045
## age         -0.0676     0.0163 -4.15 3.3e-05
## Log(scale)  -0.4590     0.2202 -2.08   0.037
## 
## Scale= 0.632 
## 
## Log Normal distribution
## Loglik(model)= -86.4   Loglik(intercept only)= -97.1
##  Chisq= 21.43 on 4 degrees of freedom, p= 0.00026 
## Number of Newton-Raphson Iterations: 6 
## n= 26
\end{verbatim}

\begin{itemize}
\tightlist
\item
  Eliminando \textbf{ecog.ps}
\end{itemize}

\begin{verbatim}
## 
## Call:
## survreg(formula = Surv(futime, fustat) ~ resid.ds + rx + age, 
##     data = base, dist = "lognormal")
##               Value Std. Error     z       p
## (Intercept) 10.3867     0.9756 10.65 < 2e-16
## resid.ds2   -0.5258     0.3288 -1.60   0.110
## rx2          0.6030     0.2984  2.02   0.043
## age         -0.0677     0.0161 -4.20 2.6e-05
## Log(scale)  -0.4609     0.2178 -2.12   0.034
## 
## Scale= 0.631 
## 
## Log Normal distribution
## Loglik(model)= -86.4   Loglik(intercept only)= -97.1
##  Chisq= 21.42 on 3 degrees of freedom, p= 8.6e-05 
## Number of Newton-Raphson Iterations: 6 
## n= 26
\end{verbatim}

\begin{itemize}
\tightlist
\item
  Eliminando \textbf{resid.ds}
\end{itemize}

\begin{verbatim}
## 
## Call:
## survreg(formula = Surv(futime, fustat) ~ rx + age, data = base, 
##     dist = "lognormal")
##               Value Std. Error     z       p
## (Intercept) 10.5449     1.0482 10.06 < 2e-16
## rx2          0.6904     0.3169  2.18   0.029
## age         -0.0765     0.0171 -4.48 7.4e-06
## Log(scale)  -0.3813     0.2189 -1.74   0.082
## 
## Scale= 0.683 
## 
## Log Normal distribution
## Loglik(model)= -87.7   Loglik(intercept only)= -97.1
##  Chisq= 18.93 on 2 degrees of freedom, p= 7.8e-05 
## Number of Newton-Raphson Iterations: 6 
## n= 26
\end{verbatim}

\begin{itemize}
\tightlist
\item
  Las variables \textbf{rx} y \textbf{age} son significativas para
  ajustar el modelo de regresión usando la distribución lognormal.
\end{itemize}

\subsecion{b. distribución log-logistic}

\begin{itemize}
\tightlist
\item
  Usando todas las variables
\end{itemize}

\begin{verbatim}
## 
## Call:
## survreg(formula = Surv(futime, fustat) ~ resid.ds + ecog.ps + 
##     rx + age, data = base, dist = "loglogistic")
##               Value Std. Error     z       p
## (Intercept) 10.4756     1.0427 10.05 < 2e-16
## resid.ds2   -0.5469     0.3530 -1.55    0.12
## ecog.ps2    -0.0346     0.3147 -0.11    0.91
## rx2          0.5957     0.3039  1.96    0.05
## age         -0.0690     0.0174 -3.97 7.2e-05
## Log(scale)  -0.9869     0.2412 -4.09 4.3e-05
## 
## Scale= 0.373 
## 
## Log logistic distribution
## Loglik(model)= -86.8   Loglik(intercept only)= -97.4
##  Chisq= 21.06 on 4 degrees of freedom, p= 0.00031 
## Number of Newton-Raphson Iterations: 6 
## n= 26
\end{verbatim}

\begin{itemize}
\tightlist
\item
  Eliminando la variable \textbf{ecog.ps}
\end{itemize}

\begin{verbatim}
## 
## Call:
## survreg(formula = Surv(futime, fustat) ~ resid.ds + rx + age, 
##     data = base, dist = "loglogistic")
##               Value Std. Error     z       p
## (Intercept) 10.4561     1.0209 10.24 < 2e-16
## resid.ds2   -0.5371     0.3404 -1.58    0.11
## rx2          0.5941     0.3026  1.96    0.05
## age         -0.0691     0.0173 -4.00 6.4e-05
## Log(scale)  -0.9892     0.2398 -4.12 3.7e-05
## 
## Scale= 0.372 
## 
## Log logistic distribution
## Loglik(model)= -86.8   Loglik(intercept only)= -97.4
##  Chisq= 21.05 on 3 degrees of freedom, p= 1e-04 
## Number of Newton-Raphson Iterations: 6 
## n= 26
\end{verbatim}

\begin{itemize}
\tightlist
\item
  Se elimina \textbf{resid.ds}
\end{itemize}

\begin{verbatim}
## 
## Call:
## survreg(formula = Surv(futime, fustat) ~ rx + age, data = base, 
##     dist = "loglogistic")
##               Value Std. Error     z       p
## (Intercept) 10.8071     1.1538  9.37 < 2e-16
## rx2          0.5911     0.3298  1.79 0.07309
## age         -0.0805     0.0187 -4.31 1.6e-05
## Log(scale)  -0.9050     0.2385 -3.79 0.00015
## 
## Scale= 0.405 
## 
## Log logistic distribution
## Loglik(model)= -88   Loglik(intercept only)= -97.4
##  Chisq= 18.62 on 2 degrees of freedom, p= 9e-05 
## Number of Newton-Raphson Iterations: 5 
## n= 26
\end{verbatim}

\begin{itemize}
\tightlist
\item
  Se elimina \textbf{rx}
\end{itemize}

\begin{verbatim}
## 
## Call:
## survreg(formula = Surv(futime, fustat) ~ age, data = base, dist = "loglogistic")
##               Value Std. Error     z       p
## (Intercept) 11.6495     1.3767  8.46 < 2e-16
## age         -0.0887     0.0225 -3.95 7.9e-05
## Log(scale)  -0.8003     0.2428 -3.30 0.00098
## 
## Scale= 0.449 
## 
## Log logistic distribution
## Loglik(model)= -89.6   Loglik(intercept only)= -97.4
##  Chisq= 15.61 on 1 degrees of freedom, p= 7.8e-05 
## Number of Newton-Raphson Iterations: 5 
## n= 26
\end{verbatim}

\begin{itemize}
\tightlist
\item
  Solo la variable \textbf{age} es significativa para ajustar el modelo
  de regresión usando la distribución loglogistic.
\end{itemize}

Evaluando los modelos con el criterio AIC y BIC

\begin{Shaded}
\begin{Highlighting}[]
\NormalTok{criterioAIC }\OtherTok{\textless{}{-}} \FunctionTok{AIC}\NormalTok{(result1, result1\_logl, result1\_logn, result2, result2\_logl, result2\_logn, result3, result3\_logl, result3\_logn, result4)}

\NormalTok{criterioBIC }\OtherTok{\textless{}{-}} \FunctionTok{BIC}\NormalTok{(result1, result1\_logl, result1\_logn, result2, result2\_logl, result2\_logn, result3, result3\_logl, result3\_logn, result4)}
\end{Highlighting}
\end{Shaded}

\begin{Shaded}
\begin{Highlighting}[]
\FunctionTok{library}\NormalTok{(kableExtra)}
\end{Highlighting}
\end{Shaded}

\begin{verbatim}
## 
## Attaching package: 'kableExtra'
\end{verbatim}

\begin{verbatim}
## The following object is masked from 'package:dplyr':
## 
##     group_rows
\end{verbatim}

\begin{Shaded}
\begin{Highlighting}[]
\NormalTok{criterioAIC}
\end{Highlighting}
\end{Shaded}

\begin{verbatim}
##              df      AIC
## result1       5 191.2886
## result1_logl  6 185.6466
## result1_logn  6 184.8166
## result2       4 189.4634
## result2_logl  5 183.6588
## result2_logn  5 182.8207
## result3       3 188.4435
## result3_logl  3 185.1017
## result3_logn  4 183.3164
## result4       2 187.5586
\end{verbatim}

\begin{Shaded}
\begin{Highlighting}[]
\NormalTok{criterioBIC}
\end{Highlighting}
\end{Shaded}

\begin{verbatim}
##              df      BIC
## result1       5 197.5791
## result1_logl  6 193.1952
## result1_logn  6 192.3652
## result2       4 194.4958
## result2_logl  5 189.9492
## result2_logn  5 189.1112
## result3       3 192.2178
## result3_logl  3 188.8760
## result3_logn  4 188.3488
## result4       2 190.0748
\end{verbatim}

\begin{Shaded}
\begin{Highlighting}[]
\NormalTok{tabla }\OtherTok{\textless{}{-}} \FunctionTok{cbind}\NormalTok{(criterioAIC, criterioBIC)}

\FunctionTok{kable}\NormalTok{(tabla[ ,}\FunctionTok{c}\NormalTok{(}\DecValTok{1}\NormalTok{, }\DecValTok{2}\NormalTok{, }\DecValTok{4}\NormalTok{)]) }\SpecialCharTok{\%\textgreater{}\%}
  \FunctionTok{kable\_styling}\NormalTok{(}\AttributeTok{full\_width =}\NormalTok{ F)}
\end{Highlighting}
\end{Shaded}

\begin{table}
\centering
\begin{tabular}{l|r|r|r}
\hline
  & df & AIC & BIC\\
\hline
result1 & 5 & 191.2886 & 197.5791\\
\hline
result1\_logl & 6 & 185.6466 & 193.1952\\
\hline
result1\_logn & 6 & 184.8166 & 192.3652\\
\hline
result2 & 4 & 189.4634 & 194.4958\\
\hline
result2\_logl & 5 & 183.6588 & 189.9492\\
\hline
result2\_logn & 5 & 182.8207 & 189.1112\\
\hline
result3 & 3 & 188.4435 & 192.2178\\
\hline
result3\_logl & 3 & 185.1017 & 188.8760\\
\hline
result3\_logn & 4 & 183.3164 & 188.3488\\
\hline
result4 & 2 & 187.5586 & 190.0748\\
\hline
\end{tabular}
\end{table}

De la tabla se observa que el modelo tres con distribución lognormal es
el que cuenta con menor BIC, pero en el otro criterio es el segundo
modelo con menor AIC, entonces, por el criterio de parsimonia se escoge
el modelo tres con distribución lognormal ya que cuenta con pocos
parámetros y las diferencias de los criterios entre los modelos no es
muy grande.

Evaluando interacción en el modelo

\begin{Shaded}
\begin{Highlighting}[]
\CommentTok{\#Interaction}

\NormalTok{resultFinal }\OtherTok{\textless{}{-}} \FunctionTok{survreg}\NormalTok{(}\FunctionTok{Surv}\NormalTok{(futime, fustat) }\SpecialCharTok{\textasciitilde{}}\NormalTok{ rx}\SpecialCharTok{*}\NormalTok{age, }\AttributeTok{dist =} \StringTok{"lognormal"}\NormalTok{, }\AttributeTok{data =}\NormalTok{ base)}
\FunctionTok{summary}\NormalTok{(resultFinal)}
\end{Highlighting}
\end{Shaded}

\begin{verbatim}
## 
## Call:
## survreg(formula = Surv(futime, fustat) ~ rx * age, data = base, 
##     dist = "lognormal")
##               Value Std. Error     z       p
## (Intercept) 10.2591     1.0583  9.69 < 2e-16
## rx2          3.9955     3.9076  1.02   0.307
## age         -0.0717     0.0174 -4.12 3.8e-05
## rx2:age     -0.0565     0.0662 -0.85   0.393
## Log(scale)  -0.3844     0.2191 -1.75   0.079
## 
## Scale= 0.681 
## 
## Log Normal distribution
## Loglik(model)= -87.2   Loglik(intercept only)= -97.1
##  Chisq= 19.8 on 3 degrees of freedom, p= 0.00019 
## Number of Newton-Raphson Iterations: 6 
## n= 26
\end{verbatim}

Se observa que la interacción en el modelo no es significativa, por lo
tanto, los efectos principales tanto de la edad como rx son
estadísticamente validos.

Evaluando un posible efecto de confusión con la variable
\textbf{resid.ds}

\begin{Shaded}
\begin{Highlighting}[]
\CommentTok{\#Confounding}

\NormalTok{resultConf }\OtherTok{\textless{}{-}} \FunctionTok{survreg}\NormalTok{(}\FunctionTok{Surv}\NormalTok{(futime, fustat) }\SpecialCharTok{\textasciitilde{}}\NormalTok{ rx }\SpecialCharTok{+}\NormalTok{ age }\SpecialCharTok{+}\NormalTok{ resid.ds, }\AttributeTok{dist =} \StringTok{"lognormal"}\NormalTok{, }\AttributeTok{data =}\NormalTok{ base)}
\FunctionTok{summary}\NormalTok{(resultConf)}
\end{Highlighting}
\end{Shaded}

\begin{verbatim}
## 
## Call:
## survreg(formula = Surv(futime, fustat) ~ rx + age + resid.ds, 
##     data = base, dist = "lognormal")
##               Value Std. Error     z       p
## (Intercept) 10.3867     0.9756 10.65 < 2e-16
## rx2          0.6030     0.2984  2.02   0.043
## age         -0.0677     0.0161 -4.20 2.6e-05
## resid.ds2   -0.5258     0.3288 -1.60   0.110
## Log(scale)  -0.4609     0.2178 -2.12   0.034
## 
## Scale= 0.631 
## 
## Log Normal distribution
## Loglik(model)= -86.4   Loglik(intercept only)= -97.1
##  Chisq= 21.42 on 3 degrees of freedom, p= 8.6e-05 
## Number of Newton-Raphson Iterations: 6 
## n= 26
\end{verbatim}

\begin{Shaded}
\begin{Highlighting}[]
\FunctionTok{summary}\NormalTok{(result3\_logn)}
\end{Highlighting}
\end{Shaded}

\begin{verbatim}
## 
## Call:
## survreg(formula = Surv(futime, fustat) ~ rx + age, data = base, 
##     dist = "lognormal")
##               Value Std. Error     z       p
## (Intercept) 10.5449     1.0482 10.06 < 2e-16
## rx2          0.6904     0.3169  2.18   0.029
## age         -0.0765     0.0171 -4.48 7.4e-06
## Log(scale)  -0.3813     0.2189 -1.74   0.082
## 
## Scale= 0.683 
## 
## Log Normal distribution
## Loglik(model)= -87.7   Loglik(intercept only)= -97.1
##  Chisq= 18.93 on 2 degrees of freedom, p= 7.8e-05 
## Number of Newton-Raphson Iterations: 6 
## n= 26
\end{verbatim}

\begin{Shaded}
\begin{Highlighting}[]
\CommentTok{\#To evaluate confounding use abs((beta\_unadjusted{-}beta\_adjusted)/beta\_unadjusted)}

\NormalTok{rxCon }\OtherTok{\textless{}{-}} \FunctionTok{abs}\NormalTok{((}\FloatTok{0.6904} \SpecialCharTok{{-}} \FloatTok{0.6030}\NormalTok{)}\SpecialCharTok{/}\FloatTok{0.6904}\NormalTok{)}\SpecialCharTok{*}\DecValTok{100}
\NormalTok{ageCon }\OtherTok{\textless{}{-}} \FunctionTok{abs}\NormalTok{((}\SpecialCharTok{{-}}\FloatTok{0.0765} \SpecialCharTok{{-}} \SpecialCharTok{{-}}\FloatTok{0.0677}\NormalTok{)}\SpecialCharTok{/{-}}\FloatTok{0.0765}\NormalTok{)}\SpecialCharTok{*}\DecValTok{100}

\NormalTok{rxCon}
\end{Highlighting}
\end{Shaded}

\begin{verbatim}
## [1] 12.65933
\end{verbatim}

\begin{Shaded}
\begin{Highlighting}[]
\NormalTok{ageCon}
\end{Highlighting}
\end{Shaded}

\begin{verbatim}
## [1] 11.50327
\end{verbatim}

\begin{Shaded}
\begin{Highlighting}[]
\CommentTok{\#Conclusion: Renal is confounding the effect of treat, so the model with both}
\CommentTok{\#treat and renal must be reported}
\end{Highlighting}
\end{Shaded}

Puesto que el cambio porcentual en las estimaciones de ambos parámetros
después de considerar el efecto de \textbf{resid.ds} es mayor al 10\%,
se puede argumentar que la variable resid.ds es una variable de
confusión.

\section{2.}

\begin{itemize}
\tightlist
\item
  Los siguientes datos se refieren a dos grupos de mujeres con ćancer de
  ovario.
\end{itemize}

\subsection{a) Ajuste un modelo Weibull a estos datos considerando como variable independiente la variable ’grupo’ que en este caso tiene dos niveles: 1 y 2. Usando las respectivas estimaciones Grafique la función hazard}

\begin{verbatim}
## 
## Call:
## survreg(formula = Surv(Tiempo, Status) ~ as.factor(Grupo), data = base2, 
##     dist = "weibull")
##                    Value Std. Error     z      p
## (Intercept)        7.015      0.354 19.81 <2e-16
## as.factor(Grupo)2 -0.829      0.418 -1.98  0.047
## Log(scale)        -0.181      0.191 -0.95  0.343
## 
## Scale= 0.834 
## 
## Weibull distribution
## Loglik(model)= -142.1   Loglik(intercept only)= -144.2
##  Chisq= 4.22 on 1 degrees of freedom, p= 0.04 
## Number of Newton-Raphson Iterations: 5 
## n= 34
\end{verbatim}

Grafico de la función hazard

\begin{verbatim}
## Loading required package: flexsurv
\end{verbatim}

\begin{verbatim}
## Warning: package 'flexsurv' was built under R version 4.1.3
\end{verbatim}

\includegraphics{Tarea2_files/figure-latex/unnamed-chunk-20-1.pdf}

\subsection{b) Con estos datos y sin usar el log-rank test pruebe la hipótesis $H_0~:S_1(t) = S_2(t)$ para todo t. Use un $\alpha = 0,05$ ¿Qué se puede concluir? ¿Es plausible la conclusión?}

PH interpretation

To translate the coefficient in an AFT model \(\( \alpha_j \)\) to that
of a PH model \(\( \beta_j \)\),

\(\( \beta_j = -\alpha_j p \)\)

where \(\( p \)\) is the shape parameter.

For this example, the coefficient is multiplied by -1, then multiplied
by the shape parameter (1/scale parameter = 1/1 for exponential model).

exp(-0.516 * -1 * 1/1) = 1.68

This is the hazard ratio comparing gender = 1 to gender = 0.

\end{document}
